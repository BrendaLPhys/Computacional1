\documentclass{article} % Tipo de documento

\usepackage[utf8]{inputenc} % Permite el uso de caracteres del Español

\usepackage[T1]{fontenc}

\usepackage{graphicx}

\usepackage{subfig}

% Carátula del Artículo  

\title{Evaluación 2}

\author{Brenda Leyva Amaya}

\date{26 de Abril, 2018}

\begin{document}

\maketitle 

\begin{center}

	\includegraphics[width=13cm]{exam.jpg}
       
\end{center}

\vspace{7.0 cm}

El sistema de Lorenz es un sistema de ecuaciones diferenciales ordinarias que fueron estudiadas primeramente por Edward Lorenz. Este sistema es notable pues presenta soluciones caóticas para algunas variaciones de sus parámetros y diferentes condiciones iniciales. Particularmente el atractor de Lorenz es una serie de soluciones caóticas del sistema de Lorenz que al ser graficadas asemejan la forma de una mariposa o de un número ocho. 

\section*{RESULTADOS}

Se han descargado los códigos correspondientes al trabajo de Geoff Boeing con el atractor de Lorenz. Este código se adapta y repite para explorar varios escenarios en distintas combinaciones de parámetros. Geoff Boeing ha puesto a disposición de cualquier persona que desee explorar el modelo este útil código en su sitio de github. 

\vspace{0.5 cm}

Este ejercicio no sólo nos ayuda a comprender a manera general la idea detrás del trabajo de Lorenz con sistemas caóticos y como su comportamiento gráfico se descubrió es muy interesante, si no que también es nuestro primer acercamiento a este tipo de gráficas y a la creación de gifs animados. 

\vspace{0.5 cm}

A continuación se muestran los principales resultados de cada uno de los escenarios estudiados con sus respectivos resultados gráficas y algunos comentarios comparativos acerca de como estos distintos parámetros se visualizan en los resultados en contraste con los otros ejemplos. 

\section{Primer ejemplo}

Se explora el modelo para los valores:

\vspace{0.5 cm}

** sigma = 10

** beta = 8/3 

** rho = 28 

\vspace{0.5 cm}

Y se obtienen las gráficas correspondientes mostradas a continuación además de la animación respectiva.

\begin{center}

	\includegraphics[width=13cm]{lorenz-attractor-3d_1.png}

Imagen que representa la solución al atractor de Lorenz de los valores ingresados para Ejemplo 1.

\end{center}


\begin{center}

	\includegraphics[width=13cm]{lorenz-attractor-phase-plane_1.png}

Imagen que representa los comportamientos gráficos de cada una de las variables con respecto al tiempo para el atractor de Lorenz de los valores ingresados para Ejemplo 1.

\end{center}

\section{Segundo ejemplo}

Se explora el modelo para los valores:

\vspace{0.5 cm}

** sigma = 28

** beta = 4 

** rho = 46.92 

\vspace{0.5 cm}

Y se obtienen las gráficas correspondientes mostradas a continuación además de la animación respectiva. En contraste con el caso anterior se ha formado una figura que consiste en dos círculos mejor definidos que presentan un grado menor de variación que el primer ejemplo.

\begin{center}

	\includegraphics[width=10cm]{lorenz-attractor-3d_2.png}

Imagen que representa la solución al atractor de Lorenz de los valores ingresados para Ejemplo 2.

\end{center}


\begin{center}

	\includegraphics[width=12cm]{lorenz-attractor-phase-plane_2.png}
       
Imagen que representa los comportamientos gráficos de cada una de las variables con respecto al tiempo para el atractor de Lorenz de los valores ingresados para Ejemplo 2.

\end{center}


\section{Tercer ejemplo}

Se explora el modelo para los valores:

\vspace{0.5 cm}

** sigma = 10

** beta = 8/3

** rho = 99.96

\vspace{0.5 cm}

Y se obtienen las gráficas correspondientes mostradas a continuación además de la animación respectiva. En este caso los comportamientos gráficos son muy distintos a los dos ejemplos previos. No se logran observar dos círculos distintivos esta vez y se define una zona de mayor actividad cargada hacia la izquierda del gráfico.


\begin{center}

	\includegraphics[width=10cm]{lorenz-attractor-3d_3.png}
       
Imagen que representa la solución al atractor de Lorenz de los valores ingresados para Ejemplo 3.

\end{center}


\begin{center}

	\includegraphics[width=13cm]{lorenz-attractor-phase-plane_3.png}

Imagen que representa los comportamientos gráficos de cada una de las variables con respecto al tiempo para el atractor de Lorenz de los valores ingresados para Ejemplo 3.

\end{center}





\end{document}